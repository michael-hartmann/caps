\documentclass[twocolumn,superscriptaddress,prb]{revtex4-1}

\usepackage[english]{babel}
\usepackage[utf8]{inputenc}
\usepackage{amssymb,amsmath}				%Mathe und Schriftsatz
\usepackage{prettyref}
\usepackage{bbm}
\usepackage{txfonts}

\usepackage[babel]{csquotes}
\usepackage[hidelinks]{hyperref}  % unbedingt als letztes Paket laden

% imaginary unit
\newcommand{\imag}{i} % imaginary unit
\newcommand{\e}{e}    % euler's number

% real/imaginary part
\renewcommand{\Re}{\mathrm{Re}}
\renewcommand{\Im}{\mathrm{Im}}

% TE, TM
\newcommand{\TE}{\mathrm{TE}}
\newcommand{\TM}{\mathrm{TM}}

\newcommand{\Plm}[2]{{\text{P}_{#1}^{#2}}}

\newcommand{\lmax}{{\ell_\text{max}}}
\newcommand{\tmax}{{\text{max}}}

\renewcommand{\vec}[1]{{\mathbf{#1}}}


\begin{document}

\title{Evaluating the integrals in the plane-sphere geometry within the multipole-approach}

\author{Michael Hartmann}
\affiliation{Universität Augsburg, Institut für Physik, 86135 Augsburg, Germany}

\date{\today}

\begin{abstract}
We briefly describe how to evaluate the integration over the wavevector $k$ in the plane-sphere geometry.
\end{abstract}

\maketitle
\section{Introduction}

In this document we briefly describe how to evaluate the integration over $k$.
After substitution, the problem is to calculate the integrals
\begin{align}
\label{eq:A}
A_{\ell_1,\ell_2,p}^{m}(\tau) &= A_0 \int_0^\infty \mathrm{d}z \, r_p \, \frac{\e^{-\tau z}}{z^2+2z} \, \Plm{\ell_1}{m}(1+z) \Plm{\ell_2}{m}(1+z), \\
\label{eq:B}
B_{\ell_1,\ell_2,p}^{m}(\tau) &= B_0 \int_0^\infty \mathrm{d}z \, r_p \, \e^{-\tau z} \, (z^2+2z) \, \Plm{\ell_1}{m}^\prime(1+z) \Plm{\ell_2}{m}^\prime(1+z), \\
\label{eq:C}
C_{\ell_1,\ell_2,p}^{m}(\tau) &= C_0 \int_0^\infty \mathrm{d}z \, r_p \, \e^{-\tau z} \, \Plm{\ell_1}{m}(1+z) \Plm{\ell_2}{m}^\prime(1+z), \\
\label{eq:D}
D_{\ell_1,\ell_2,p}^{m}(\tau) &= (-1)^{\ell_1+\ell_2+1} C_{\ell_2\ell_1,p}^{m},
\end{align}
where
\begin{align}
A_0 &= (-1)^{\ell_2+m} \Lambda_{\ell_1,\ell_2}^m m^2 \e^{-\tau}, \\
B_0 &= (-1)^{\ell_2+m+1} \Lambda_{\ell_1,\ell_2}^m \e^{-\tau}, \\
C_0 &= (-1)^{\ell_2+m} \Lambda_{\ell_1,\ell_2}^m \imag m \e^{-\tau},
\end{align}
and $\tau \equiv 2nT$. The prefactor $\Lambda_{\ell_1,\ell_2}^m$ is defined as
\begin{equation}
\Lambda_{\ell_1,\ell_2}^m = -2 \frac{N_{\ell_1}^m N_{\ell_2}^m}{\sqrt{\ell_1 (\ell_1+1) \ell_2 (\ell_2+1)}},
\end{equation}
and
\begin{equation}
N_\ell^m = \sqrt{\frac{2\ell+1}{2} \frac{(\ell-m)!}{(\ell+m)!}}
\end{equation}
is the normalizing factor of spherical harmonics.
The integrals \eqref{eq:A}--\eqref{eq:D} are identical to (6.11)--(6.14) in
\cite{hartmann} after the substitution $z = x/\tau$.


\section{Eliminating the derivatives}

Using the recurrence relation \eqref{appendix:dPlm} we can express derivatives
of associated Legendre polynomials as a sum of associated Legendre polynomials:
\begin{align}
\frac{A_{\ell_1,\ell_2,p}^{m}}{A_0} =& \mathcal{I}_{\ell_1,\ell_2,p}^{m} \\
\frac{B_{\ell_1,\ell_2,p}^{m}}{B_0} =& \frac{(\ell_1+1)(\ell_1+m)(\ell_2+1)(\ell_2+m)}{(2\ell_1+1)(2\ell_2+1)} \mathcal{I}_{\ell_1-1,\ell_2-1,p}^{m} \nonumber \\
                                     & -\frac{\ell_1(\ell_1-m+1)(\ell_2+1)(\ell_2+m)}{(2\ell_1+1)(2\ell_2+1)} \mathcal{I}_{\ell_1+1,\ell_2-1,p}^{m} \nonumber \\
                                     & - \frac{(\ell_1+1)(\ell_1+m)\ell_2(\ell_2-m+1)}{(2\ell_1+1)(2\ell_2+1)} \mathcal{I}_{\ell_1-1,\ell_2+1,p}^{m} \nonumber \\
                                     & + \frac{\ell_1(\ell_1-m+1)\ell_2(\ell_2-m+1)}{(2\ell_1+1)(2\ell_2+1)} \mathcal{I}_{\ell_1+1,\ell_2+1,p}^{m} \\
\frac{C_{\ell_1,\ell_2,p}^{m}}{C_0} =&  \frac{\ell_2 (\ell_2-m+1)}{2\ell_2+1} \mathcal{I}_{\ell_1,\ell_2+1,p}^{m} - \frac{(\ell_2+1)(\ell_2+m)}{2\ell_2+1} \mathcal{I}_{\ell_1,\ell_2-1,p}^{m}
\end{align}
where we have defined
\begin{equation}
\mathcal{I}_{\ell_1,\ell_2,p}^{m}(\tau) = \int_0^\infty \mathrm{d}z \, r_p \, \frac{\e^{-\tau z}}{z^2+2z} \, \Plm{\ell_1}{m}(1+z) \Plm{\ell_2}{m}(1+z) \,.
\end{equation}
So, we reduced the problem of calculating the integrals $A,B,C,D$ to the
problem of calculating the integral $\mathcal{I}$. The integral is symmetric
with respect to $\ell_1$ and $\ell_2$
\begin{equation}
\mathcal{I}_{\ell_1,\ell_2,p}^{m} = \mathcal{I}_{\ell_2,\ell_1,p}^{m} \,.
\end{equation}

Using Gaunt coefficients a product of associated Legendre polynomials can be
expressed as a sum of associated Legendre polynomials \cite{gaunt}
\begin{equation}
\Plm{\ell_1}{m}(x) \Plm{\ell_2}{m}(x) = a_0 \sum_{q=0}^{q_\tmax} \tilde a_q \Plm{\ell_1+\ell_2-2q}{2m}(x) \,.
\end{equation}
Thus we can express the integral $\mathcal{I}_{\ell_1,\ell_2,p}^{m}$ as
\begin{align}
\nonumber
\mathcal{I}_{\ell_1,\ell_2,p}^{m}(\tau) &= \int_0^\infty \mathrm{d}z \, r_p \, \frac{\e^{-\tau z}}{z^2+2z} \, \Plm{\ell_1}{m}(1+z) \Plm{\ell_2}{m}(1+z) \\
& = a_0 \sum_{q=0}^{q_\tmax} \tilde a_q \, \mathcal{K}_{\ell_1+\ell_2-2q}^{2m}(1+z)
\end{align}
where we have defined
\begin{equation}
\mathcal{K}_{\nu,p}^m = \int_0^\infty \mathrm{d}z \, r_p \frac{\e^{-\tau z}}{z^2+2z} \Plm{\nu}{2m}(1+z) \,.
\end{equation}
The advantage is that for a scattering matrix $\mathcal{M}^m(\xi)$ we only have
to evaluate $\mathcal{O}(\lmax)$ instead of $\mathcal{O}(\ell_\tmax^2)$
integrals.

\section{Shape of integrand}

To estimate the shape and the maximum of $\mathcal{K}_{\nu,p}^m$ we assume that
the Fresnel coefficient $r_p$ varies slowly and ignore it.

\subsection{Large values}
For large values of $z$ we can approximate the integrand of
$\mathcal{K}_{\nu,p}^m$ using \eqref{appendix:Plm_gg}
\begin{equation}
k(z) = r_p \frac{\e^{-\tau z}}{z^2+2z} \Plm{\nu}{2m}(1+z) \overset{z \gg 1}{\approx} c \e^{-\tau z} z^\nu ,
\end{equation}
where $c$ is a constant. We find that the maximum of the integrand is
approximately at $z_\tmax \approx \nu/\tau$. As
\begin{equation}
\Plm{\nu}{2m}(1+z_\tmax) \approx \frac{(2\nu)!}{2^\nu \, \nu! \, (\nu-2m)!} z_\tmax^\nu
\end{equation}
the integrand at $z_\tmax$ is approximately
\begin{equation}
k(z_\tmax) \approx \frac{(2\nu)!}{2^\nu \nu! (\nu-2m)!} \e^{-\tau z_\tmax} z_\tmax^\nu \,.
\end{equation}

\subsection{Small values}
For small values of $z$ we can approximate the integrand of
$\mathcal{K}_{\nu,p}^m$ using \eqref{appendix:Plm_1}
\begin{equation}
k(z) = r_p \frac{\e^{-\tau z}}{z^2+2z} \Plm{\nu}{2m}(1+z) \overset{z \ll 1}{\approx} c \e^{-\tau z} z^{m-1}
\end{equation}
where $c$ is a constant. The maximum of the integrand is approximately at $z_\tmax \approx m/\tau$ and the
integrand at $z_\tmax$ is approximately
\begin{equation}
k(z_\tmax) \approx \frac{(\nu+2m)!}{(2m)! \, (\nu-2m)!} \frac{\e^{-\tau z_\tmax}}{z^2+2z} \left(\frac{z}{2}\right)^m \,.
\end{equation}

\appendix

\section{Formulae}


\subsection{Associated Legendre polynomials}

We define associated Legendre polynomials for $x>1$ as
\begin{equation}
\Plm{\ell}{m}(x) = \frac{(-\imag)^m}{2^\ell \ell!} \left(x^2-1\right)^{m/2} \frac{\mathrm{d}^{\ell+m}}{\mathrm{d}x^{\ell+m}} \left(x^2-1\right)^\ell \,.
\end{equation}
The associated Legendre polynomials fullfil many recurrence relations, but we will only need this one:
\begin{equation}
\label{appendix:dPlm}
\frac{\mathrm{d}}{\mathrm{d}x} \Plm{\ell}{m}(x) = \frac{1}{x^2-1} \left[
\frac{\ell(\ell-m+1)}{2\ell+1} \Plm{\ell+1}{m}(x)
-\frac{(\ell+1)(\ell+m)}{2\ell+1} \Plm{\ell-1}{m}(x)
\right]
\end{equation}
For large arguments $x\gg 1$ the associated Legendre polynomials may be approximated by
\begin{equation}
\label{appendix:Plm_gg}
\Plm{\ell}{m}(x) \approx (-\imag)^m \frac{(2\ell)!}{2^\ell \, \ell! \, (\ell-m)!} x^\ell,
\end{equation}
for $x\sim1$ they may be approximated by
\begin{equation}
\label{appendix:Plm_1}
\Plm{\ell}{m}(x) \approx \frac{(\ell+m)!}{m! \, (\ell-m)!} \left(\frac{x-1}{2}\right)^{m/2} \,.
\end{equation}

\addcontentsline{toc}{chapter}{Bibliography}


\begin{thebibliography}{99}

\bibitem{hartmann}
  Michael Hartmann,
  \emph{Negative Casimir entropies in the plane–sphere geometry}, master thesis, 2014

\bibitem{gaunt}
  Yu-lin Xu,
  \emph{Fast evaluation of Gaunt coefficients: recursive approach}, Journal of Computational and Applied Mathematics, 1997

\end{thebibliography}

\end{document}
