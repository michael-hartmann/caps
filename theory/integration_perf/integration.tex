\documentclass[10pt]{article}

\usepackage[english]{babel}
\usepackage{setspace}
\usepackage[T1]{fontenc}						%Europäischer Schriftsatz
\usepackage{a4wide}					 				%Bessere Ausnutzung der DinA4 größe ohne Randbemerkungen, auch a4 möglich
\usepackage[utf8]{inputenc}
\usepackage{amssymb,amsmath}				%Mathe und Schriftsatz
\usepackage{prettyref}
\usepackage{bbm}
\usepackage{txfonts}
\usepackage[format=plain,margin=1.5cm,small]{caption} %Legenden: mergin versetzt beitseitig nach innen, format=plain verhindert einrücken (einfacher blocksatz)

\usepackage[url=false,isbn=false,eprint=false,natbib=true,style=numeric]{biblatex}

\usepackage[babel]{csquotes}
\usepackage[hidelinks]{hyperref}  % unbedingt als letztes Paket laden

% imaginary unit
\newcommand{\imag}{i} % imaginary unit
\newcommand{\e}{e}    % euler's number

% real/imaginary part
\renewcommand{\Re}{\mathrm{Re}}
\renewcommand{\Im}{\mathrm{Im}}

% TE, TM
\newcommand{\TE}{\mathrm{TE}}
\newcommand{\TM}{\mathrm{TM}}

\newcommand{\Plm}[2]{{\text{P}_{#1}^{#2}}}

\newcommand{\sep}{\quad}

\newcommand{\F}{\ensuremath{\mathcal{F}}}

\newcommand{\Id}{\mathbbm{1}}
\newcommand{\qmax}{{q_\text{max}}}
\newcommand{\lmax}{{\ell_\text{max}}}

\renewcommand{\vec}[1]{{\mathbf{#1}}}


\begin{document}

\section{Integration for perfect reflectors}

We want to compute the integrals
\begin{align}
\label{eq:A}
A_{\ell_1\ell_2,p}^{(m)}(\tau) &= (-1)^{\ell_2+m} \, m^2 \, \e^{-\tau} \int_0^\infty \mathrm{d}z \, r_p \, \frac{\e^{-\tau z}}{z^2+2z} \, \Plm{\ell_1}{m}(1+z) \Plm{\ell_2}{m}(1+z), \\
\label{eq:B}
B_{\ell_1\ell_2,p}^{(m)}(\tau) &= (-1)^{\ell_2+m+1} \, \e^{-\tau} \int_0^\infty \mathrm{d}z \, r_p \, \e^{-\tau z} \, (z^2+2z) \, \Plm{\ell_1}{m}^\prime(1+z) \Plm{\ell_2}{m}^\prime(1+z), \\
\label{eq:C}
C_{\ell_1\ell_2,p}^{(m)}(\tau) &= \imag m (-1)^{\ell_2+m} \, \e^{-\tau} \int_0^\infty \mathrm{d}z \, r_p \, \e^{-\tau z} \, \Plm{\ell_1}{m}(1+z) \Plm{\ell_2}{m}^\prime(1+z), \\
\label{eq:D}
D_{\ell_1\ell_2,p}^{(m)}(\tau) &= (-1)^{\ell_1+\ell_2+1} C_{\ell_2\ell_1,p}^{(m)}
\end{align}
for perfect reflectors, where $\tau \equiv 2nT=2\xi$. The Fresnel coefficients
for perfect reflectors, $r_\TM=-r_\TE=1$, are independent of $\vec k$ and
can be pulled in front of the integration. The integrals \eqref{eq:A}--\eqref{eq:D}
are identical to the integrals (6.11)--(6.14) in MASTERTHESIS after substitution $z = x/\tau$.

Using Gaunt coefficients a product of associated Legendre polynomials can be
expressed as a sum of associated Legendre polynomials:
\begin{equation}
\Plm{n}{m}(x) \Plm{\nu}{\mu}(x) = a_0 \sum_{q=0}^\qmax \tilde a_q  \Plm{n+\nu-2q}{m+\mu}(x)
\end{equation}
Every integral can be reduced to a sum of integrals of the type
\begin{equation}
\mathcal{J}_\nu^{2m}(\tau) = \int_0^\infty \mathrm{d}z \, \frac{\e^{-\tau z}}{z^2+2z} \Plm{\nu}{2m}\left(1+z\right).
\end{equation}


\subsection{Integral $A_{\ell_1\ell_2,p}^{(m)}(\tau)$}

\begin{align}
\nonumber
A_{\ell_1\ell_2,p}^{(m)}(\tau) &= \overbrace{(-1)^{\ell_2+m} \, r_p \, m^2 \, \e^{-\tau}}^{\equiv A_0} \int_0^\infty \mathrm{d}z \, \frac{\e^{-\tau z}}{z^2+2z} \, \Plm{\ell_1}{m}(1+z) \Plm{\ell_2}{m}(1+z) \\
\nonumber
&= A_0 \, a_0 \sum_{q=0}^\qmax \tilde a_q \int_0^\infty \mathrm{d}z \, \frac{\e^{-\tau z}}{z^2+2z} \, \Plm{\ell_1+\ell_2-2q}{2m}(1+z) \\
&= A_0 \, a_0 \sum_{q=0}^\qmax \tilde a_q \, \mathcal{J}_{\ell_1+\ell_2-2q}^{2m}(\tau)
\end{align}


\subsection{Integral $B_{\ell_1\ell_2,p}^{(m)}(\tau)$}

\begin{align}
\nonumber
&B_{\ell_1\ell_2,p}^{(m)}(\tau) = \overbrace{(-1)^{\ell_2+m+1} \, r_p \, \e^{-\tau}}^{\equiv B_0} \int_0^\infty \mathrm{d}z \, \e^{-\tau z} \, (z^2+2z) \, \Plm{\ell_1}{m}^\prime(1+z) \Plm{\ell_2}{m}^\prime(1+z) \\
\nonumber
&\sep = \frac{B_0}{(2\ell_1+1)(2\ell_2+1)} \int_0^\infty \mathrm{d}z \, \frac{\e^{-\tau z}}{z^2+2z} \left[ (\ell_1+1)(\ell_1+m)\Plm{\ell_1-1}{m}(1+z) - \ell_1(\ell_1-m+1)\Plm{\ell_1+1}{m}(1+z) \right] \\
\nonumber
& \sep\sep \times \left[ (\ell_2+1)(\ell_2+m)\Plm{\ell_2-1}{m}(1+z) - \ell_2(\ell_2-m+1)\Plm{\ell_2+1}{m}(1+z) \right] \\
\nonumber
&\sep = \frac{B_0}{(2\ell_1+1)(2\ell_2+1)} \Bigg[ \\
\nonumber
   &\sep\sep\sep +a_0                      \sum_{q=0}^\qmax                        (\ell_1+1)(\ell_1+m)(\ell_2+1)(\ell_2+m) \, \tilde a_q \, \mathcal{J}_{\ell_1+\ell_2-2-2q}^{2m}(\tau) \\
\nonumber
   &\sep\sep\sep -a_0^\prime               \sum_{q=0}^{\qmax^\prime}               (\ell_1+1)(\ell_1+m)\ell_2(\ell_2-m+1)   \, \tilde a_q^\prime \, \mathcal{J}_{\ell_1+\ell_2-2q}^{2m}(\tau) \\
\nonumber
   &\sep\sep\sep -a_0^{\prime\prime}       \sum_{q=0}^{\qmax^{\prime\prime}}       \ell_1(\ell_1-m+1)(\ell_2+1)(\ell_2+m)   \, \tilde a_q^{\prime\prime} \, \mathcal{J}_{\ell_1+\ell_2-2q}^{2m}(\tau) \\
   &\sep\sep\sep +a_0^{\prime\prime\prime} \sum_{q=0}^{\qmax^{\prime\prime\prime}} \ell_1(\ell_1-m+1)\ell_2(\ell_2-m+1)     \, \tilde a_q^{\prime\prime\prime} \, \mathcal{J}_{\ell_1+\ell_2+2-2q}^{2m}(\tau) \Bigg]
\end{align}


\subsection{Integral $C_{\ell_1\ell_2,p}^{(m)}(\tau)$}

\begin{align}
\nonumber
&C_{\ell_1\ell_2,p}^{(m)}(\tau) = \overbrace{\imag m (-1)^{\ell_2+m} \, r_p \, \e^{-\tau}}^{\equiv C_0} \int_0^\infty \mathrm{d}z \, \e^{-\tau z} \, \Plm{\ell_1}{m}(1+z) \Plm{\ell_2}{m}^\prime(1+z) \\
\nonumber
%&\sep\overset{\eqref{appendix:dPlm}}{=} C_0 \int_0^\infty \mathrm{d}z \, \e^{-\tau z} \, \Plm{\ell_1}{m}(1+z) \frac{(\ell_2+1)(\ell_2+m) \Plm{\ell_2-1}{m}(1+z) - \ell_2(\ell_2-m+1)\Plm{\ell_2+1}{m}(1+z)}{(2\ell_2+1)(z^2+2z)} \\
&\sep\overset{\eqref{appendix:dPlm}}{=} C_0 \int_0^\infty \mathrm{d}z \, \frac{\e^{-\tau z}}{z^2+2z} \, \Plm{\ell_1}{m}(1+z) \, \left( \frac{(\ell_2+1)(\ell_2+m)}{2\ell_2+1} \Plm{\ell_2-1}{m}(1+z) - \frac{\ell_2(\ell_2-m+1)}{2\ell_2+1} \Plm{\ell_2+1}{m}(1+z) \right) \\
&\sep = \frac{C_0}{2\ell_2+1} \left(
    a_0 (\ell_2+1)(\ell_2+m) \sum_{q=0}^\qmax \tilde a_q \mathcal{J}_{\ell_1+\ell_2-1-2q}^{2m}(\tau)
    - a_0^\prime \ell_2(\ell_2-m+1) \sum_{q=0}^{\qmax^\prime} \tilde a_q^\prime \mathcal{J}_{\ell_1+\ell_2+1-2q}^{2m}(\tau) \right)
\end{align}


\subsection{Integral $\mathcal{J}_\nu^{2m}(\tau)$}

\begin{align}
\nonumber
\mathcal{J}_\nu^{2m}(\tau) &= \int_0^\infty \mathrm{d}z \, \e^{-\tau z} (z^2+2z)^{-1} \Plm{\nu}{2m}(1+z) \\
\nonumber
&\overset{\eqref{appendix:Plm}}{=} \int_0^\infty \mathrm{d}z \, \e^{-\tau z} (z^2+2z)^{-1} (-1)^{2m} \left(1-1-2z-z^2\right)^m \frac{\mathrm{d}^{2m}}{\mathrm{d}z^{2m}} \Plm{\nu}{}(1+z) \\
\nonumber
&\overset{\eqref{appendix:Pn}}{=} (-1)^m \int_0^\infty \mathrm{d}z \, \e^{-\tau z} (z^2+2z)^{m-1} \frac{\mathrm{d}^{2m}}{\mathrm{d}z^{2m}} \sum_{k=0}^\nu \binom{\nu}{k} \binom{-\nu-1}{k} \left(-\frac{1}{2}\right)^k z^k \\
\nonumber
&\overset{\eqref{appendix:binom}}{=} (-1)^m \int_0^\infty \mathrm{d}z \, \e^{-\tau z} (z^2+2z)^{m-1} \sum_{k=2m}^\nu \binom{\nu}{k} \binom{\nu+k}{k} \left(+\frac{1}{2}\right)^k \frac{k!}{(k-2m)!} z^{k-2m} \\
&= (-1)^m \int_0^\infty \mathrm{d}z \, \e^{-\tau z} (z^2+2z)^{m-1} \sum_{k=2m}^\nu \frac{(k+\nu)!}{2^k k! (k-2m)! (\nu-k)!} z^{k-2m}
\end{align}

\appendix

\section{Formulae}

\subsection{Binomial coefficient}
\begin{equation}
\label{appendix:binom}
\binom{-\alpha}{k} = (-1)^k \binom{\alpha+k-1}{k}
\end{equation}


\subsection{Legendre polynomials}

\begin{equation}
\label{appendix:Pn}
\Plm{n}{}(x) = \sum_{k=0}^n \binom{n}{k} \binom{-n-1}{k} \left(\frac{1-x}{2}\right)^k
\end{equation}


\subsection{Associated Legendre polynomials}

\begin{equation}
\label{appendix:Plm}
\Plm{\ell}{m}(x) = (-1)^m (1-x^2)^{m/2} \frac{\mathrm{d}^m}{\mathrm{d}x^m} \Plm{\ell}{}(x)
\end{equation}

\begin{equation}
\label{appendix:dPlm}
\frac{\mathrm{d}}{\mathrm{d}x} \Plm{\ell}{m}(x) = \frac{1}{(2\ell+1) \, (1-x^2)} \left[ (\ell+1)(\ell+m)\Plm{\ell-1}{m}(x) - \ell(\ell-m+1)\Plm{\ell+1}{m}(x) \right]
\end{equation}

\clearpage
\addcontentsline{toc}{chapter}{Bibliography}
\printbibliography

\end{document}
