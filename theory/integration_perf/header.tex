\documentclass[
%	pdftex,%              PDFTex verwenden
	a4paper,%             A4 Papier
	twoside,%             Zweiseitig/Einseitigtotocnumbered,%    Literaturverzeichnis nummeriert einf�gen
%	idxtotoc,%            Index ins Verzeichnis einf�gen
	index=totoc,
%	halfparskip,%         Europ�ischer Satz mit abstand zwischen Abs�tzen
	parskip=half,
 	%chapterprefix,%       Kapitel anschreiben als Kapitel
	headsepline,%         Linie nach Kopfzeile
%	footsepline,%         Linie vor Fusszeile
	10pt,%                Gr��ere Schrift, besser lesbar am bildschrim
	BCOR5mm,%							5mm Abstand Rand
	fleqn,%								linksb�ndig abgesetzte Formeln
  openany%,%              keine leeren seiten zwischen kapiteln
  %ngerman
]{scrbook}

\usepackage[english]{babel}
\usepackage{setspace}
\usepackage{a4wide}					 				%Bessere Ausnutzung der DinA4 gr��e ohne Randbemerkungen, auch a4 m�glich
\usepackage[T1]{fontenc}						%Europ�ischer Schriftsatz
\usepackage[utf8]{inputenc}
\usepackage{graphicx}								%Einbindung von Graphiken
\usepackage{amssymb,amsmath}				%Mathe und Schriftsatz
\usepackage{prettyref}
\usepackage{diagbox}
\usepackage{xspace}
\usepackage{colortbl}
\usepackage{enumitem}
\usepackage{bbm}
\usepackage[format=plain,margin=1.5cm,small]{caption} %Legenden: mergin versetzt beitseitig nach innen, format=plain verhindert einr�cken (einfacher blocksatz)
\usepackage{subcaption}
\usepackage{ulem}

%Graphikerweiterungen f�r eps-Graphiken
\usepackage{subcaption}
\usepackage{epsfig}
\usepackage{rotating}
\usepackage{braket}

% index generation
\usepackage{makeidx}
\makeindex
 
% 'list of notations' generation
\usepackage{nomencl}
\makenomenclature


\setkomafont{sectioning}{\normalfont\bfseries}
\setkomafont{captionlabel}{\normalfont\bfseries}
%\setkomafont{pagehead}{\normalfont\itshape}
\setkomafont{descriptionlabel}{\normalfont\bfseries}

% Caption-Stil
% \setlength{\captionindent}{3cm}
% \renewcommand{\captionlabelfont}{\bfseries \sffamily}

% Tabellen:

% gr\"{o}{\ss}ere Zeilenh\"{o}he
\setlength{\extrarowheight}{0.1cm}

% Neue Spaltenstile, f\"{u}r Dezimalzahlen
\newcolumntype{1}{D{.}{.}{1.13}}
% und grau hinterlegte Tabellenzellen
\newcolumntype{G}{>{\columncolor[gray]{0.8}}c}

% Abk\"{u}rzung f\"{u}r graue Tabellenzelle
\newcommand{\GS}[1]{\multicolumn{1}{G}{#1}}

% Farbe zwischen Doppellinien aus \hhline:
\doublerulesepcolor{white}


\usepackage{txfonts}

% Tabelleneinbindung
\usepackage{array}

% EPS-Grafiken:
\graphicspath{{images/}}
% \setlength{\intextsep}{5ex plus 1ex minus 1ex}

% Erstellung von Index
\usepackage{makeidx}
\makeindex
\definecolor{ForestGreen}{rgb}{0, 0.545, 0} %definiert dunkleres gr�n mit name in rgb (werte zwischen 0 und 1)
\usepackage{hyphenat} %Silbentrennung verhindern mit: \nohyphens{***}

\usepackage{wrapfig}

\usepackage[url=false,isbn=false,eprint=false,natbib=true,style=numeric]{biblatex}

\usepackage[babel]{csquotes}
\bibliography{lit.bib}  % bindet lit.bib ein
\usepackage[hidelinks]{hyperref}  % unbedingt als letztes Paket laden

% generalized vector
%\renewcommand{\vec}[1]{{\mathbf{#1}}}

% imaginary unit
\newcommand{\imag}{i}

% euler's number
\newcommand{\e}{e}

% column vector
%\newcommand{\cvec}[3]{\begin{pmatrix} #1 \\ #2 \\ #3 \end{pmatrix}}

% Re and Im
\renewcommand{\Re}{\mathrm{Re}}
\renewcommand{\Im}{\mathrm{Im}}

% complex conjugate
\newcommand{\conjugate}[1]{#1^*}

% abs function
\newcommand{\abs}[1]{\left|#1\right|}

% soll gleich sein
\newcommand{\se}{\stackrel{!}{=}}

% TE, TM
\newcommand{\TE}{\mathrm{TE}}
\newcommand{\TM}{\mathrm{TM}}

\newcommand{\wignerd}{\mathrm{d}}

\newcommand{\PE}{\text{E}}
\newcommand{\PM}{\text{M}}

\newcommand{\kb}{k_\text{B}}

\newcommand{\Ylm}[1]{Y_{#1}}
\newcommand{\Plm}[2]{{\text{P}_{#1}^{#2}}}
\newcommand{\barPlm}[2]{{\bar{\text{P}}_{#1}^{#2}}}

\newcommand{\Nlm}[1]{\text{N}_{#1}}

\newcommand{\sep}{\quad}

\newcommand{\SI}{\text{SI}}
\newcommand{\F}{\ensuremath{\mathcal{F}}}
\renewcommand{\L}{\ensuremath{\mathcal{L}}}

\newcommand{\Id}{\mathbbm{1}}
\newcommand{\qe}{e}
\newcommand{\me}{m_\qe}
\newcommand{\nee}{n_\qe}
\newcommand{\qmax}{{q_\text{max}}}
\newcommand{\lmax}{{\ell_\text{max}}}


\DeclareMathOperator{\Tr}{Tr}

\renewcommand{\log}{\text{ln}}

\newcommand{\Li}{\operatorname{Li}}
