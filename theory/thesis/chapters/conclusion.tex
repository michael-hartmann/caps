\chapter{Conclusion and outlook}

In this master thesis, we have studied the Casimir effect in the plane--sphere
geometry, in particular with respect to the effect of negative entropies. We
have shown that the Maxwell equations in vacuum are equivalent to the
source-free vector Helmholtz equation and introduced the plane wave basis and
the multipole basis as solutions. Moreover, we have presented the solutions to
the scattering problems of electromagnetic waves at a planar, homogenous
interface and at a sphere, where the metallic properties of plane and sphere
are either described by the Drude model, the plasma model or the model of
perfect reflectors. With the Fresnel and Mie coefficients as well as the
solutions of the Helmholtz equation at hand, we have applied the scattering
approach to the plane--plane and to the plane--sphere geometry. Although
equivalent to the matrix elements of \textsc{Durand} et al. \cite{Durand,
ThermalCasimirEffect}, our matrix elements of the scattering matrix in the
plane--sphere geometry avoid Wigner D-matrix elements. After changing to scaled
quantities the free energy for perfect reflectors depends only on temperature
and separation. Also, we have considered analytically the contribution to the
free energy of the Matsubara frequency $\xi=0$, because this special case
raises numerical problems. The software implementation of the formulas reveals
several serious numerical problems, e.g. the problem of computing the
determinant of a matrix whose elements differ by hundreds of orders of
magnitude. We have discussed these problems and presented solutions for a fast
and stable software implementation.

We have compared our numerical results with the proximity force approximation (PFA)
that links the unknown free interaction energy of arbitrary geometries to the
well-known energy of the plane--plane geometry. We have shown that indeed the
PFA becomes a good approximation in the limit of small separations. As
entropies obtained using the PFA are always positive, we have conjectured that the
effect of negative entropies vanishes for small separations. This assumption
was strengthened by the study of the large--distance limit. We have derived an
analytical expression for the free energy in the limit of large separations and
have shown that the negative entropies are linked with a change of polarization
within a round trip. For low temperatures we showed that the entropy as a
function of the separation has a minimum and probably tends to zero for small
separations. As the matrix elements of the scattering matrix become independent
of the particular properties of the material in the high temperature limit, we
have studied this limit for perfect reflectors as well as for Drude mirrors. We
have ruled out several suggestions for the free energy at small separations and
proofed the validity of an expansion by \textsc{Bimonte} and \textsc{Emig}. At
last, we have shown that negative entropies exist over a wide range of parameters.
In accordance with previous assumptions, we have found that negative entropies
disappear for small separations.  While the entropy for large separations is
negative for (scaled and thus dimensionless) temperatures $T\lesssim1.5$, the
entropy has a minimum for $T\approx0.927$ and $R/L\approx0.207$. The smaller
the separation the lower is the temperature at which the entropy becomes
positive. This means that the effect of negative entropies is more evident for
large separations. In contrast to this, at the same time the Casimir effect
becomes less pronounced for large separations.

This master thesis has covered the main aspects of the Casimir effect in the
plane--sphere geometry for perfect reflectors. Further studies could focus on
effects due to plasma oscillations and finite conductivity of the metallic
mirrors. This adds two more length scales and makes the parameter space more
complicated, yet more interesting. Moreover, for a more accurate description it
is also necessary to consider effects due to corrugations of plane and sphere
\cite{Lambrecht:CasimirScatteringApproach}.
