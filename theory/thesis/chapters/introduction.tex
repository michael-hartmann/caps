\chapter{Introduction}

In 1948 Hendrik Casimir considered two parallel, perflectly conducting plates
in vacuum at temperature $T=0$ and predicted an attracting force
\cite{Casimir1948}. This force was experimentally verified in 1956 by
\textsc{Derjaguin}, \textsc{Abrikosova} and \textsc{Lifshitz}
\cite{DerjaguinAbrikosovaLifshits1956}, and in 1958 by \textsc{Sparnaay}
\cite{Sparnaay1958}. The accuracy of Casimir experiments has been drastically
improved over the last decades and several experiments claim to achieve a
precision at the 1\% level \cite{2012IJMPA2760013L}. Experiments are usually
carried out using a metallic sphere and a metallic plate at ambient
temperature. In particular, the plane--sphere geometry avoids misalignments
that render accurate meassurements in the plane--plane configuration difficult.

Since the Casimir effect is a manifestation of vacuum fluctuations in the
mesoscopic world, it has relations to many open physical questions. As the
Casimir force is the dominant force between electrically neutral bodies at
micron or sub-micron distances, it plays an important role in the search for
new hypothetical forces predicted by unified theories like string theory
\cite{Onofrio2006}. Moreover, the Casimir effect is linked to the theory of
gravitation and in particular to the cosmological constant. All energy
gravitates and thus zero point fluctuations are expected to contribute to the
stress--energy tensor in Einstein's field equations \cite{jaffe2005casimir}.
In fact, several cosmological observations like the discovery of the
acceleration of the universe \cite{1998AJ....116.1009R} suggest that the energy
density of vacuum is non-zero. However, the measured value of the cosmological
constant and the estimation due to zero point fluctuations differ by about 120
orders of magnitude \cite{hobson2006general}. Also, Casimir forces are closely
related with van der Waals forces and may be interpreted as retarded van der
Waals forces \cite{2000PhRvA..61f2107K}.

Many different theoretical approaches to the Casimir effect have been developed
or revised in the last decade: Experiments are usually compared with results
obtained using the proximity force approximation (PFA). The PFA is an
approximation that links arbitrary geometries to the simple plane--plane
geometry. Although no error estimates exist, the PFA is believed to become
accurate when the separation between the objects becomes small
\cite{2011PhRvD84j5031F}. Further methods include worldline numerics
\cite{2006PhRvD..74d5002G, 2003hep.th...11168M} or approximations based on
classical ray optics \cite{2004PhRvL..92g0402J}. Yet another powerful method is
the scattering approach \cite{CasimirWithinScatteringTheory} which links the
scattering operators in vacuum with the Casimir free energy.

Moreover, negative entropies are found for some geometries and parameters in
the Casimir effect. Negative entropies, for example, occur in the plane--plane
geometry for metals described by the Drude model. In addition, this effect also
occurs in the plane--sphere geometry even for perfect reflectors, thus
suggesting a geometrical origin of negative entropies
\cite{2014arXiv1405.0311M}. This is in general not a problem, since the Casimir
free energy is an interaction energy and does not describe the entire physical
system \cite{2014arXiv1405.0311M}. However, the origin of negative entropies is
still unclear\footnote{There is now a better understanding for negative Casimir
entropies, see \cite{PhysRevE.91.033203,PhysRevE.92.042125}.}.

In this thesis, we will apply the scattering approach to the Casimir effect in
the plane--sphere geometry at finite temperature and study the effect of
negative entropies for perfect reflectors. We will show that the Maxwell
equations in vacuum are equivalent to the vector Helmholtz equation, present
the solutions to the scattering of plane waves at a sphere and a plane, and
derive the matrix elements of the scattering operator. This enables us to
compare the proximity force approximation (PFA) with exact numerical results.
In the limit of large separations we will derive an analytical expression for
the Casimir free energy and study the origin of negative entropies. For low
temperatures we investigate and compare free energy and entropy with those
obtained within the PFA. In the high temperature limit the evaluation of the
scattering formula becomes notably simpler which enables us to study smaller
separations and check various suggested expressions for the free energy.
Finally, we investigate negative entropies for arbitrary separations and
temperatures. This will help us to gain a deeper understanding of the effect of
negative entropies.
